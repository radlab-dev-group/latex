\section{Login}

Podstawą działania Przeglądarki i Eksploratora informacji są newsy pojawiające się codziennie w mediach. 
Zdefiniujmy zatem pojedynczy, dowolny news jako $n$ w dniu $d$ za pomocą: $n(d) = \textrm{news}(d)$ 
zaś konkretny news w dniu oznaczmy za pomocą $n_{i}(d) = news_{i}(d)$, wszystkie newsy w danym dniu 
oznaczmy za pomocą $N(d)$, zaś liczba dni, to $z$:
\begin{equation}
    \begin{aligned}
        n_{i}(d) = & \: \textrm{news}_{i}(d), \: i \in \left<0, k\right>, k = |N(d)| \\
        N(d) = & \left\{ n_{1}(d), ..., n_{k}(d) \right\}, \: d \in \left<d_{1}, ..., d_{z} \right>
    \end{aligned}
\end{equation}
a konkretny news w konkretnym dniu oznaczymy za pomocą $n_{i}(d_{j})$, gdzie $i$ wskazuje na newsa, 
a $j$ na dzień. W ten sposób można zdefiniować zbiór danych, który będzie analizowany w dalszej części. 
Czyli newsy, które pojawiają się codziennie na różnych stronach internetowych (newsowych). Z naszej perspektywy 
informacja rozsiana jest w wielu newsach, zatem informacja $I$ integruje w jednym miejscu różne $n_{i}$ 
i co najważniejsze $I$ zawsze posiada datę początkową i końcową, w odróżnieniu od daty $n_i(d)$ -- news 
pojawia się w danym dniu, zaś informacja rozsiana jest po wielu dniach. Wyróżniamy dwa rodzje informacji 
(jednak w dalszej części, dla prostoty definicji, wprowadzimy pomocniczy typ informacji): 
ciągłą -- $CI$ (lokalną) i nieciągłą -- $NCI$ (globalną). Zarówno $CI$ jak i $NCI$ zawierają 
dzień rozpoczęcia $d_{b}$ oraz dzień zakończenia $d_{e}$. Czyli:
\begin{equation} \label{eq:def_info}
    \begin{aligned}
        CI_{d_{b}}^{d_{e}} \subset & 
            \left\{ 
                n_{i}(d_{b}), \: 
                n_{i}(d_{b+1}), \: 
                n_{i}(d_{b+2}), \: 
                ..., \: 
                n_{i}(d_{e}) 
            \right\}, \: e \geq b \\
        NCI_{d_{b}}^{d_{e}} \subset & 
            \left\{ 
                CI_{d_{b}}^{d_{e'}}, 
                ..., 
                CI_{d_{b'}}^{d_{e}} 
            \right\}, \: e \geq b' \geq b, \: e \geq e' \geq b \\
        & |CI| \leq |N(d)|, \: |NCI| \leq |CI|
    \end{aligned}
\end{equation}
Gdzie $d_{b}$ ($b$ - begin) oznacza datę początkową, a $d_{e}$ ($e$ -- end) datę końcową informacji $I$. 
W ten sposób zdefiniowaliśmy czym jest news, czym jest informacja oraz jak je rozróżniamy.