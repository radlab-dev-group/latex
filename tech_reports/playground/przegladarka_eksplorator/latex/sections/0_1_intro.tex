\section{Intro}

Od niedawna na naszym \textit{playgroundzie} działają dwa rozwiązania Przeglądarka Informacji oraz Eksplorator informacji. 
To narzędzia, które pomagają spojrzeć na pojawiające się newsy z szerszej perspektywy -- pozwalają analizować 
jak dana wiadomość wpasowuje się w ogólny trend informacyjny. Lub z drugiej strony: jak pojawiające się newsy 
tworzą bańki informacyjne. Działanie Przeglądarki i Kreatora nastawione jest na analizę informacji, nie pojedynczych 
wiadomości, a wiadomość wpasowana jest w informację. Przykładowe ogólne informacje: \textit{Wojna na Ukrainie}, 
\textit{Pogoda}, \textit{Wybory prezydenckie w Polsce}. Można powiedzieć, że informacja w pewnym stopniu definiuje tematykę, 
która z nią jest powiązana. Jednak z innej strony, informacja może być lokalna np. \textit{Druga tura wyborów prezydenckich}, 
która pojawia się w określonym czasie i kończy w momencie, kiedy przestanie się o niej pisać. 
Do analizy \textit{lokalnych} i \textit{globalnych} informacji, służą właśnie wspomniane rozwiązania,
a jako \textit{źródło zasilania} metod danymi, we wspomnianych rozwiązaniach, 
wykorzystany został \textit{Strumień Aktualności} z \textit{playgroundu}, 
w którym analizujemy informacje pojawiające się na ponad 20 polskich i zagranicznych stronach newsowych.
